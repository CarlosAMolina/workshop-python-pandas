%%%%%%%%%%%%%%%%%%%%%%%%%%%%%%%%%%%%%%%%%
% Beamer Presentation
% LaTeX Template
% Version 1.0 (10/11/12)
%
% This template has been downloaded from:
% http://www.LaTeXTemplates.com
%
% License:
% CC BY-NC-SA 3.0 (http://creativecommons.org/licenses/by-nc-sa/3.0/)
%
%%%%%%%%%%%%%%%%%%%%%%%%%%%%%%%%%%%%%%%%%

\documentclass{beamer}

\mode<presentation> {

\usetheme{Madrid}

%\setbeamertemplate{footline} % To remove the footer line in all slides uncomment this line
%\setbeamertemplate{footline}[page number] % To replace the footer line in all slides with a simple slide count uncomment this line

\setbeamertemplate{navigation symbols}{} % To remove the navigation symbols from the bottom of all slides uncomment this line
}

\usepackage{graphicx} % Allows including images
\usepackage{booktabs} % Allows the use of \toprule, \midrule and \bottomrule in tables
\definecolor{links}{HTML}{2A1B81}
\hypersetup{colorlinks,linkcolor=,urlcolor=links}


\graphicspath{ {./imagenes/} }

%----------------------------------------------------------------------------------------
% TITLE PAGE
%----------------------------------------------------------------------------------------

\title[Pandas \& testing]{Pandas \& testing} % The short title appears at the bottom of every slide, the full title is only on the title page
\author{Carlos A Molina}
\date{25 Marzo 2021}

\begin{document}

\begin{frame}
\titlepage % Print the title page as the first slide
\end{frame}

\begin{frame}
\frametitle{Índice}
\tableofcontents % Throughout your presentation, if you choose to use \section{} and \subsection{} commands, these will automatically be printed on this slide as an overview of your presentation
\end{frame}

%----------------------------------------------------------------------------------------
% PRESENTATION SLIDES
%----------------------------------------------------------------------------------------

%------------------------------------------------
\section{Introducción} % Sections can be created in order to organize your presentation into discrete blocks, all sections and subsections are automatically printed in the table of contents as an overview of the talk
%------------------------------------------------

\begin{frame}

\frametitle{Python. Origen de su nombre}

  \begin{figure}
    \centering
    \href{http://www.python.org/doc/faq/general/\#why-is-it-called-python}
      {\includegraphics[width=0.9\textwidth]{brian.jpeg}}
  \end{figure}

\end{frame}

\begin{frame}

\frametitle{Extract Transform Load}

  \begin{figure}
    \centering
    \includegraphics[width=\textwidth]{etl.png}
  \end{figure}

\end{frame}

\section{Demostración}

\begin{frame}
\frametitle{Demostración}
\begin{columns}[c] % The "c" option specifies centered vertical alignment while the "t" option is used for top vertical alignment

\column{.5\textwidth}
  \begin{figure}
    \centering
    \includegraphics[width=0.7\textwidth]{olla.png}
  \end{figure}

\column{.45\textwidth}
\begin{itemize}
\item SQL VS Pandas
\item Pickle
\item Testing
\end{itemize}


\end{columns}
\end{frame}

\section{Recursos}

\begin{frame}
\frametitle{Recursos}
\begin{itemize}
\item BBVA Investment Funds\\
\href{https://www.bbva.es/personas/productos/fondos/por-tipo-de-activo.html}{https://www.bbva.es}
\item Docker MySQL\\
\href{https://hub.docker.com/\_/mysql/}{https://hub.docker.com}
\item ETL\\
\href{https://www.inetsoft.com/business/solutions/etl\_definition\_advantages\_and\_disadvantages/}{https://www.inetsoft.com}
\item MySQL exportar e importar base de datos\\
\href{https://www.tutorialspoint.com/mysql/mysql-database-export.htm}{https://www.tutorialspoint.com}\\
\href{https://phoenixnap.com/kb/import-and-export-mysql-database}{https://phoenixnap.com}
\item LaTeX template\\
\href{https://www.latextemplates.com/template/beamer-presentation}{https://www.latextemplates.com}
\end{itemize}
\end{frame}
\begin{frame}
\frametitle{Recursos}
\begin{itemize}
\item Python. Origen de su nombre\\
\href{http://www.python.org/doc/faq/general/\#why-is-it-called-python}{http://www.python.org}
\item SQLAlchemy\\
\href{https://docs.sqlalchemy.org/en/14/core/engines.html}{https://docs.sqlalchemy.org}
\end{itemize}
\end{frame}


%------------------------------------------------

\begin{frame}
\Huge{\centerline{Muchas gracias}}
\end{frame}

%----------------------------------------------------------------------------------------

\end{document} 
